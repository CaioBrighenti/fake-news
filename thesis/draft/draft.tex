% !TeX program = pdfLaTeX
\documentclass[12pt]{article}
\usepackage{amsmath}
\usepackage{graphicx,psfrag,epsf}
\usepackage{enumerate}
\usepackage{natbib}
\usepackage{textcomp}
\usepackage[hyphens]{url} % not crucial - just used below for the URL
\usepackage{hyperref}
\providecommand{\tightlist}{%
  \setlength{\itemsep}{0pt}\setlength{\parskip}{0pt}}

%\pdfminorversion=4
% NOTE: To produce blinded version, replace "0" with "1" below.
\newcommand{\blind}{0}

% DON'T change margins - should be 1 inch all around.
\addtolength{\oddsidemargin}{-.5in}%
\addtolength{\evensidemargin}{-.5in}%
\addtolength{\textwidth}{1in}%
\addtolength{\textheight}{1.3in}%
\addtolength{\topmargin}{-.8in}%

%% load any required packages here




\setlength\parindent{24pt}

\begin{document}


\def\spacingset#1{\renewcommand{\baselinestretch}%
{#1}\small\normalsize} \spacingset{1}


%%%%%%%%%%%%%%%%%%%%%%%%%%%%%%%%%%%%%%%%%%%%%%%%%%%%%%%%%%%%%%%%%%%%%%%%%%%%%%

\if0\blind
{
  \title{\bf An Interpretable Approach to Fake News Detection}

  \author{
        Caio Brighenti \\
    Department of Computer Science, Colgate University\\
      }
  \maketitle
} \fi

\if1\blind
{
  \bigskip
  \bigskip
  \bigskip
  \begin{center}
    {\LARGE\bf An Interpretable Approach to Fake News Detection}
  \end{center}
  \medskip
} \fi

\bigskip
\begin{abstract}
The text of your abstract. 200 or fewer words.
\end{abstract}

\noindent%
{\it Keywords:} 
\vfill

\newpage
\spacingset{1.45} % DON'T change the spacing!

\section{Introduction}
\label{sec:intro}

~~~~~Propaganda has long been a tool of political influence, but in
recent years it has taken a new online form: fake news. Fake news, once
a buzzword on the internet, is now at the center of global politics, one
of the most common bigram in the lexicon of United States president
Donald Trump. After the term gained prominence in Trump's presidential
campaign in 2016, it exploded into public conciousness, earning the
distinction of Webster-Collins' ``Word of the Year'' in 2017.\footnote{}
As the current presidential race unfolds, fake news has returned to the
center of the conversation, with major social media companies facing
scrutiny of their misinformation policy. This phenomenon is also not a
distinctly American problem--investigate reporting both during and after
the 2018 Brazilian president election demonstrated that more than XX\%
of news articles shared on the popular messaging service WhatsApp were
fake news.\footnote{}~\\
\hspace*{0.333em}\hspace*{0.333em}\hspace*{0.333em}\hspace*{0.333em}\hspace*{0.333em}Fake
news is not only politically significant but also dangerously tempting.
Studies have shown false content propagates faster through social media
than real content.\footnote{} Blatantly false or exaggerated rhetoric
can even lead to violent action, as demonstrated by the ``Pizzagate''
incident in which a man stormed a D.C. pizzeria with an AR-15, having
been convinced by false and unverified information that a pedophile ring
operated out of the restaurant's basement.\footnote{} Fake news can also
be incredibly easy to create. In 2019, a group of researchers at the
Allen Institute for Artificial Intelligence published text generation
model able to produce fake news.\footnote{} In a troubling conclusion,
the researchers found that state-of-the-art fake news detection systems
struggled more with identifying fake news produced by their systems than
actual fake news.\footnote{} Fake news is thus easy to create, spreads
quickly, and is hard to detect, a dangerous combination making it a
serious threat to civic society.\\
\hspace*{0.333em}\hspace*{0.333em}\hspace*{0.333em}\hspace*{0.333em}\hspace*{0.333em}Given
the danger that fake news poses, machine learning and natural language
processing researchers have devoted significant attention to the problem
of fake news classification. However, previous attempts at fake news
classification overwhelmingly rely on highly complex models suffering
from the ``black box'' problem. As a result, these models lack
interpretability and do not allow us to reach new conclusions about the
nature of fake news. In order to begin closing this gap, this research
adopts an interpretable approach, with the overall objective of a
producing a fake news classification model with comparable accuracy to
state of the art models without compromising interpretability.\\
\hspace*{0.333em}\hspace*{0.333em}\hspace*{0.333em}\hspace*{0.333em}\hspace*{0.333em}PARAGRAPH
SUMMARIZING FINDINGS

\section{Prior Work}
\label{sec:verify}

~~~~~Since the 2016 U.S. presidential election, fake news has been a
frequent topic of natural language processing research. There are
countless examples of papers approaching fake news classification or
closely related problems, but from slightly different angles. This
section summarizes the prior work in fake news detection, clarifying the
different categories of methods. In general, previous works in fake news
detection differ in three major ways: 1) the scale of the predicted
variable, the information used as features, and the type of model. With
respect to scale, any fake news detection model falls into one of three
levels of granularity: 1) claim level, 2) source level, and 3) article
level. These levels of analysis describe the response variable being
predicted.\\
\hspace*{0.333em}\hspace*{0.333em}\hspace*{0.333em}\hspace*{0.333em}\hspace*{0.333em}Claim
level approaches attempt to determine whether a given claim, usually one
or several sentences, is true or intentionally misleading. \footnote{Examples
  of claim-level approaches include \ldots{}} Given that claim-level
approaches must make a judgement based on only a short amount of text,
researchers often adopt a fact-checking strategy This strategy, also
known as ``truth discovery,'' assumes that a sentence's claims can be
gramatically isolated and checked against a database of established
claims.\footnote{strube p.~2} A natural application for claim-level
models is social media, most commonly Twitter, where little is known
about the author and only a very limited amount of text is available.
However, claim-level approaches have serious limitations, often
struggling with the complex sentences journalists or other writers
typically employ.\footnote{strube 2} Aditionally, they rely entirely on
a complete knowledge base, which must be constantly expanded and
updated, clearly a difficult task.\\
\hspace*{0.333em}\hspace*{0.333em}\hspace*{0.333em}\hspace*{0.333em}\hspace*{0.333em}Source
level approaches attempt to classify whether a speaker or entire news
source consitently publish misinformation. The intuition behind these
approaches is that speakers or sources that have published
misinformation in the past are likely to continue to do so. An example
of a source-level approach is the popular browser extension ``BS
Detector,''\footnote{} which classifies articles on a fine-grained scale
of veracity by checking the source's status in a database of news
sources and their reliability. A source's history of misinformation can
also be used as a predictor in claim-level or article level approaches.
Kirilin and Strube, for instance, create \emph{Speaker2Credit}, a metric
of speaker credibility, and show how it can improve the performance of
fake news detection models when used as an input.\footnote{Kirilin and
  Strube}\\
\hspace*{0.333em}\hspace*{0.333em}\hspace*{0.333em}\hspace*{0.333em}\hspace*{0.333em}Article
level approaches have received the most attention in the work on fake
news detection. This is logical, given that fake news tends to take the
form of articles, peddling misinformation in the article text while
posing as a legitimate source. Article-level approaches also benefit
from a rich list of predictors to choose from, including not only the
article's content but all relevant metadata. Kai Shu et. al, for
instance, build an article-level model using linguistic and visual
components of the article content, the social context around
it---including information on the user that posted it, the post itself,
responses to it, and the social network of the poster---, as well as
spatiotemporal information capturing when and where the article and
responses were to it were posted from.\footnote{Shu et al}\\
\hspace*{0.333em}\hspace*{0.333em}\hspace*{0.333em}\hspace*{0.333em}\hspace*{0.333em}The
work of Shu et. al is an example of the overwhelming number predictors
available to researchers working in fake news detection, resulting in a
diversity of approaches with respect to feature selection. Melanie Tosik
et. al, for instance, employ only hand-crafted features capturing the
similarity between an article's title and text in a two-stage ensemble
classifier modeling whether an article's body agrees with its
headline.\footnote{Tosik et al} Sonam Tripathi and Tripti Sharma
demonstrate the effectiveness of parts of speech tagging---also known as
grammatical tagging---in document classification problems, the general
category of natural language processing that article-level fake news
detection falls under.\footnote{tripathi} Ramy Baly et al.~employ a
breadth of features to model factuality and bias of news sources, using
features covering the content of articles, the source's Wikipedia and
Twitter pages, the structure of the URL, and the source's web traffic.
\footnote{}~\\
\hspace*{0.333em}\hspace*{0.333em}\hspace*{0.333em}\hspace*{0.333em}\hspace*{0.333em}Independent
of level of analysis or choice of predictors, researchers overwhelmingly
choose to use complex deep neural networks. Ajao et. al, for instance,
use a ``hybrid of convolutional neural networks and long-short term
recurrent neural network models'' to classify Tweets as true or false
based on their text content.\footnote{ajao et al} The dominance of deep
learning approaches is visible in an extensive survey on fake news
detection done by Ray Oshikawa and Jing Qian. \footnote{Oshikawa} The
pair's section on machine learning models dedicates a total of three
sentences to ``Non-Neural Network Models,'' compared to seven
paragraphs. focusing on neural networks.\\
\hspace*{0.333em}\hspace*{0.333em}\hspace*{0.333em}\hspace*{0.333em}\hspace*{0.333em}While
deep learning approaches can produce highly accurate models that
consistently succeed in identifying misinformation, they also suffer
from a lack of interpretability. This is often referred to as the
``black-box'' problem, meaning that the inputs and outputs of these
models are perfectly clear, but the steps that the model takes to reach
the output are completely invisible. This has been identified as a
limitation of the the work on fake news detection thus far.\footnote{shu,
  O'Brien} Oshikawa and Qian, at the conclusion of their extensive
survey, declare that ``we need more logical explanation for fake news
characteristics,'' highlighting the need for models that can teach us
something about fake news.\\
\hspace*{0.333em}\hspace*{0.333em}\hspace*{0.333em}\hspace*{0.333em}\hspace*{0.333em}Approaches
that focus on interpretability are rare, but do exist. From the deep
learning perspective, Nicole O'Brien et al.~employ post-hoc variable
importance to their text-based deep learning model, identifying the
words that are most predictive of fake and real news.\footnote{O'Brien
  et al} Their approach, however, does not interpret the results, but
instead merely demonstrates the feasiblity of the technique.
Furthermore, this method reveals only information about \emph{specific}
words, as opposed to \emph{types} of words. More applicable is the work
of researchers who both use features that describe the semantic
properties of the text in general, use interpretable models, and
extensively document their results. The best examples of this type of
work are the works of Benajmin Horne et al.~and Mauricio Gruppi et al.,
both of which employ features capturing the complexity, style, and
psychology of fake news, and display precisely how fake and real news
differs in each of the variables used.\footnote{Horne, Gruppi}

\section{Methodology}

~~~~~This paper seeks to contribute to the small literature of
interpretable fake news detection, by following the methodology of Horne
et al.~and Gruppi et al., leveraging features that describe textual
properties of fake news, applying non-neural network models and focusing
heavily on interpretation of results. This section details the precise
methodology employed, discussing the dataset used, the feature
engineering process, outlier removal, and models applied.\\
\hspace*{0.333em}\hspace*{0.333em}\hspace*{0.333em}\hspace*{0.333em}\hspace*{0.333em}There
are many datasets freely available for fake news detection, but many
suffer quality limitations. Oshikawa and Qiang outline 12 requirements
for a quality fake news dataset, expanding a 9-point list originally by
Rubin et al.: 1. Availability of both truthful and deceptive instances;
2. Digital textual format acessibility; 3. Verifiability of ground
truth; 4. Homogeneity in lengths; 5. Homogeneity of writing mattes; 6.
Predefined timeframe; 7. The manner of news delivery; 8. Pragmatic
concerns; 9. Language and culture; 10. Easy to create from raw data; 11.
Fine-grained truthfulness; 12. Various sources or publishers.\footnote{Oshikawa,
  Rubin} While no dataset currently exists that meets all 12 criteria,
the article-level \emph{FakeNewsNet} (FNN) dataset meets most.\footnote{FNN}\\
\hspace*{0.333em}\hspace*{0.333em}\hspace*{0.333em}\hspace*{0.333em}\hspace*{0.333em}FNN
is an article-level dataset that includes the title and body of each
article (2), each of which have been profesionally fact-checked by
Politifact or Gossicop(3). FNN provides both political and celebrity
news articles, but this paper chooses to use only the political articles
in order to maintain a roughly consistent corpus ()

\section{Results}

\section{Discussion}

\section{Future Work}

\bibliographystyle{agsm}
\bibliography{bibliography.bib}

\end{document}
